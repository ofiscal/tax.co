\documentclass[12pt]{article}
\usepackage{amssymb}
\usepackage{amsfonts}
\usepackage{amsmath}
\usepackage[nohead]{geometry}
\usepackage[singlespacing]{setspace}
\usepackage[bottom]{footmisc}
\usepackage{indentfirst}
\usepackage{endnotes}
\usepackage{graphicx}%
\usepackage{rotating}
\usepackage{booktabs}
\usepackage{hyperref}
\usepackage{multirow}
\usepackage{makecell}
\usepackage{natbib}
\usepackage[title]{appendix}
\usepackage{lscape}
\usepackage{pdflscape}
\geometry{left=1in,right=1in,top=1.00in,bottom=1.0in}
\begin{document}

\title{A Static Microsimulation of the Colombian Tax System
  \thanks{We thank Universidad Javeriana for its financial support of this project.
    We are also grateful to Catalina Herr\'{a}n for her detailed description of the Colombian tax system,
    and to seminar participants at Universidad Javeriana for their useful comments on this draft.}}
\author{Jeffrey Benjamin Brown, Liliana Heredia, Oliver Pardo, Luis Carlos Reyes, and David Su\'{a}rez
  \thanks{Observatorio Fiscal,
    Department of Economics, Pontificia Universidad Javeriana, Bogot\'{a}, Colombia.
    Telephone:\ 57-1-3208320,
    e-mail: \textit{ofiscal@javeriana.edu.co}}}

\maketitle
\begin{abstract}
  For over at least three decades,
  Colombia has undergone major tax reform roughly every eighteen months.
  However, estimates of the distributional impact of changes to the tax code are not systematically produced by any branch of government,
  nor by independent groups,
  despite the constitutional mandate that the tax system be progressive.
  We develop a static microsimulation of the Colombian tax system
  that allows us to calculate effective tax rates
  for the entire income distribution.
  We are able to estimate differences in effective tax rates by
  (among other things) city, gender, and socioeconomic stratum.
  The software that implements the model allows for
  easy updates as changes to the tax code are proposed.
 \end{abstract}

\textbf{JEL Classification Codes:}

\textbf{Keywords:} Taxation, Tax Microsimulation, Colombia
\pagebreak%
\doublespacing

\section{Introduction}

In this paper, we develop a microsimulation model of the Colombian tax system
with the twin goals of
accurately understanding the distributional effects of taxation
in a developing country characterized by its high levels of inequality
\iffalse
    [[AN ECONOMIST:
    and informality?
    I suspect the answer is that no,
    the simulation currently has nothing to say about informality.
    While we do have a ``making pension contributions'' variable,
    we don't even report it in the final output tables,
    and the word ``formal'' only shows up twice in the simulation code,
    in the very early stages, as if we thought we might use it but never did.
    ]],
\fi
and of making this information available to the public and policymakers
in a way that is easy to understand and use in the formulation of public policy.

Few policymakers, voters, or even economists understand both
the economics of taxation
and the intricacies of tax law.
By making reasonable assumptions about tax incidence,
carefully modeling tax rules that apply to individual tax payers,
and using data from surveys and tax returns,
microsimulation analysis can boil down this complex subject to
bar charts showing the effects of proposed policies
on the after-tax income of people from various demographics and income groups.

The simplicity of the model's output is crucial.
For ideas to be circulated widely and have an effect on public policy,
they need to be clear and to the point.

The next section describes the broad outlines of taxation in Colombia.
Section 3 lists the assumptions we make about tax incidence.
Section 4 describes the data we use.
Section 5 explains in detail how we calculate effective tax rates,
and shows effective rates of taxation for different income groups.
Section 6 concludes.

\section{Institutional framework}

\iffalse
    [[SOMEONE:
    We should update all the numbers in this section,
    including the tables it references.]]
\fi

There are national, departmental, and municipal taxes in Colombia.
Departments and municipalities do not have the same degree of autonomy
that states and cities do in federal countries like the United States:
whereas American states can create and levy taxes
without the intervention of the federal government,
in Colombia they cannot.
National, departmental and municipal taxes
are all created directly by Congress.

The difference between national and local taxes
is that departments and municipalities can vary
some parameters of the taxes that Congress authorizes them to levy.
For instance, municipalities can tax the profits of
all commercial activity occurring within their boundaries,
and they can choose the tax rate as well as exemptions to the tax base.
But the tax rate cannot exceed the maximum allowed by Congress,
and municipalities do not have legal authority to, say,
create a personal income tax in their jurisdiction.

This arrangement makes it easier to create a model
of all the national and local taxes in the country.
Even though, in principle,
there are 32 departmental and 1132 municipal tax codes,
the differences among them are relatively limited.

Tax intake is low in Colombia
relative to other Latin American countries and to countries in the OECD.
According to the OECD,
tax intake in Colombia was around 19.6 \% of GDP in 2016,
whereas on average it was
22\% in Latin America and
34\% in the OECD.
\iffalse
    [[reference]]
\fi

Table \ref{tab:taxcol} breaks down tax revenue for the year 2017.
National taxes are the bulk of tax revenue,
accounting for 84.9\% of the total.
Interestingly, municipalities collect roughly three times as much in taxes as departments,
even though departments are the larger administrative unit:
municipal taxes accounted for 11.3\% of tax intake,
while departments only received 3.8\%.

Three quarters of tax revenue come from four types of taxes:
the value added tax (27.6\%),
corporate income tax (25.3\%),
payroll taxes (16.2\%),
and the personal income tax (5.9\%).
These are all national taxes.
The fifth largest source of tax revenue is the industry and commerce tax,
which is a tax on corporate income collected by municipalities
that accounts for 4.2\% of revenue.
Unlike in many developed countries, in Colombia
the personal income tax is a smaller source of revenue than corporate income taxes.

Municipal property taxes on land and real estate follow,
making up 3.7\% of revenue.
Another 3.7\% comes from a much-criticized but easy to collect national tax on financial transactions,
which is said to keep much of the population away from the formal financial system.
Import tariffs account for 2.2\% of tax revenue
-- in stark contrast with the days of import substitution industrialization,
when roughly a third of tax revenue came from tariffs.
A tax on personal and corporate wealth
that seems to come and go with every tax reform
accounted for 2\% of revenue in 2017.
Another 5.9\% of revenue comes from a variety of specific and ad valorem taxes on goods,
including a national sales tax for restaurant consumption and other items,
and various national and local taxes on
gasoline, alcohol, tobacco, and vehicles.
A host of other small taxes make up the remaining 3.3\% of tax intake.

Any analysis of taxation in Colombia must take into account that
roughly half of the workforce is informal,
which limits the collection of payroll and personal income taxes.
The assumptions of our model take this into account.

\section{Assumptions}

We are interested in calculating
the entire tax burden of individuals belonging to different income groups,
so that we can produce estimates of
how progressive proposed changes to the tax code will be.
This requires making assumptions about
the incidence of various taxes. We assume:

\begin{itemize}
  \item The personal income tax is fully borne by households.
  \item Payroll taxes are borne entirely by workers.
  \item Excise taxes
    (including the VAT,
    the tax on financial transactions,
    sales tax, and tariffs)
    are borne by the consumers of the goods and services in question.
  \item Corporate income tax is fully borne by the owners of capital.
  \item Property taxes are borne by
    the owners of land or real estate.
\end{itemize}

With the exception of the property tax assumption,
which is our own,
\iffalse
    [[LUIS:
        It is, right?]]
\fi
these assumptions are all made as well by
the Congressional Budget Office of the United States,
as described in \citet{salanie}.
\iffalse
    [[AN ECONOMIST::
        Who is salamie? That string only appears once in the entire document.
        --jbb]]
\fi
(The CBO does not model property taxes,
because in the US those are not federal.)

Appendix A describes the relevant taxes in detail.

\section{Data}
Our data come from the
\textit{Encuesta nacional de presupuestos de los hogares} (ENPH),
a nationally representative survey of 87,201 households
that was carried out between July 2016 and July 2017.
The ENPH includes a wealth of data on income and expenditures,
which we use to calculate effective tax rates for
people of different levels of income in different regions of the country.

\iffalse
[[SOMEONE:
    Double-check each of these figures
    against the table it comes from.]]
[[JEFF:
    Since the underlying data is cleaner now,
    either regenerate the table
    or verify that the Makefile regenerates it.]]
\fi

Table \ref{tab:ygender} shows some descriptive statistics of the sample.
The average household has 3.3 members and a monthly income of 2.1 million COP.
Labor income accounts for 74\% of the total,
capital income for 11.5\%,
dividend income for 0.05\%,
pension income for 10\%,
government transfers for 0.7\%,
private transfers for 3.5\%
and extraordinary income for 0.5\%.
The average household has 3.3 members,
and the average income of female-headed households
is only 73\% of the average income of male-headed ones,
despite being of essentially the same size.

Table \ref{tab:ydecile} disaggregates these statistics by income decile.
Labor income is only 53\% of income for individuals in the first decile,
with private (27\%) and government transfers (11.7\%)
being important sources of income.
As income increases, reliance on transfers falls quickly,
and income from capital and dividends slowly increases.
50\% households in the first decile are female-headed,
and their share declines as income rises.
Only 29.6\% of households in the top decile are headed by women.

Table \ref{tab:ycity} shows the statistics for the main cities,
with Bogota and Medellin reporting the highest incomes, as would be expected.

It is worth noting that the incomes reported in the survey
probably do not include the employee share of social security payments,
which we calculate on the basis of tax rules
to be able to compute effective tax rates.

\section{Effective Tax Rates}

Let $\mathcal{P}$ be a partition of the set of households.
For example, $\mathcal{P}$ could be the set of household percentiles.
We define the effective tax rate $\tau_P$
for a subset $P\in\mathcal{P}$ of households as:
\begin{equation}
\tau_P = \frac{\sum_{h\in P}T_{h}} {\sum_{h\in P}Y_h}
\end{equation}
\iffalse [[JEFF: Verify: is that even what we do?
    I (Jeff) thought we compute an average of ratios,
    not a ratio of averages.]]
\fi
where:
\begin{equation}
Y_h = \sum_{i\in h}\sum_{s\in INC}y_{is}
\end{equation}
\begin{equation}
T_h = \sum_{j\in TAXES}t_{hj}
\end{equation}
\begin{equation}
  INC = \left\{ \begin{array}{cc}
    net \; labor \;income,\ capital \;income,\  \\
    dividend \;income,\ payroll \;income,\ pension \;income
  \end{array} \right\}
\end{equation}
\begin{equation}
  TAXES = \left\{ \begin{array}{cc}
    consumption \;tax,\ capital \;tax,\  payroll \;tax, wealth \;tax, \\
    personal\ income \;tax, financial\ \;tax, property \;tax
  \end{array} \right\}
\end{equation}

\iffalse
PITFALL (DONE):
It's unfortunate to have to specify ``tax'' and ``income''
after each of the terms in the preceding two equations,
but otherwise they are confusible --
e.g. ``personal could be personal income or a tax on it;
``labor'' and ``payroll'' could be income or taxes on that income; etc.
\fi

We define
individual $j$'s effective (average) tax rate $t_j$ before transfers as
\begin{equation}
  t_j=\frac
  {T_j}
  {Y_j}
\end{equation}
where $T_j$ and $Y_j$ for individual $j$ is computed analogously to
$T_h$ and $Y_h$ for household $h$.

\iffalse
[[AN ECONOMIST:
    Should $corporate \;tax$, described below, be in the numerator
    in the definition of $t_j$?
    --jbb]]

[[JEFF:
    BLOCKED -- might be addressed once I've renamed taxes and consumptions.

    It's confusing that, for instance,
    ``consumption'' refers to the \textit{tax} on consumption,
    rather than consumption itself.
    I'm thinking $tp_{consumption}$ would be better,
    indicating ``taxes paid on consumption''.
    (`t` would be more terse than `tp`,
    but usually `t` means a tax rate, not a peso amount.)
    --jbb]]
\fi

(The) $consumption \;tax$
is payments made for value added taxes on consumption,
as well as the national sales tax
and the beer, liquor, tobacco, and vehicle taxes:
\begin{equation}
consumption \;tax = \sum^{N}_{n=1} \tau_nP_nX_n
\end{equation}
where $P_n$ is the price before taxes of good $n$,
$X_n$ is the number of units of $n$ consumed,
and $\tau_n$ is the tax rate rate that applies to good $n$.
The ENPH has data on after-tax spending,
i.e. $(P_n+\tau_n)X_n$,
which we multiply $\frac{\tau_n}{1+\tau_n}$
and sum over all goods to obtain $consumption \;tax$.
We obtain $\tau_n$ from the tax code.
Sometimes there is more than one excise tax on a good,
in which case $\tau_n$ is the sum of
the rates corresponding to the various taxes.
In the case of taxes on fuel,
we assume that not just household fuel purchases but also
a certain percentage of spending on public transportation
(which can be specified in the simulation)
\iffalse
[[LUIS:
    I don't think it can, actually.
    Did we promise that before writing the microsimulation,
    or am I forgetting something?
    --jbb]]
\fi
is taxed at this rate.
In cases where inputs other than fuel are taxed, we do the same
(see the appendix for details).

$corporate$ (income) $tax$ is individual $j$'s share of
the taxes paid by firms of which he is a stockholder:
\begin{equation}
corporate = \sum^{K}_{k=1} \tau_k\theta^j_k\Pi_k
\end{equation}
where $\theta^j_k$ is the share of firm $k$ owned by $j$,
$\tau_k$ is firm $k$'s effective tax rate,
and $\Pi_k$ is firm $k$'s profits before taxes.
The ENPH has data on total earnings from dividends.
(\citet{juliana} also have administrative data
\iffalse
    [[AN ECONOMIST:
        Who is juliana? That string only appears cited in this document,
        but the citation is never defined. --jbb]]
\fi
for the top 1\% of income earners),
which we take to be $\sum^{K}_{k=1} \theta^j_k\Pi_k(1-\tau_k)$.
To obtain $corporate \;tax$ we multiply total earnings from dividends by
$\frac{\tau_k}{1-\tau_k}$,
assuming that $\tau_k=\tau \; \forall k$,
where $\tau$ is the effective tax rate on capital
obtained from national accounts by \citet{banrep}.
\iffalse
[[AN ECONOMIST:
    I'm unaware of any place in the code that does this,
    and pretty sure that's because there isn't one.
    Maybe Luis Carlos Reyes or David Suarez was computing it by hand.
    --jbb]]

[[ AN ECONOMIST:
    What document is that ``banrep'' citation supposed to cite?
    It's not defined anywhere.
    --jbb]]
\fi

$payroll$ is the sum of employee and employer contributions to social security payments:
\begin{equation}
  payroll(Y) = \begin{cases}
    \left\{ \begin{array}{cc}
      pension \;contributions(Y) +
      health \;care \;contributions(Y) + \\
      family \;compensation \;fund \;contributions(Y) + \\
      adult \;education \;contributions(Y) +
      child \;welfare \;contributions(Y)
    \end{array} \right\}
    & Y \geq \underline{w} \\
    0 & Y < \underline{w}
    \end{cases}
\end{equation}
where $\underline{w}$ is the minimum wage,
and where each contribution is a function of $Y$,
labor income excluding the employer share's of payroll taxes
but including the employee's share.
$Y$ can (with some work) be determined from the ENPH.
\iffalse JEFF: RESUME HERE (mid-paragraph) \fi
The functions of $Y$ in \ref{payroll_tax} are calculated by
applying the rules described in the appendix.
When $Y < \underline{w}$
(which is the case for about half of the labor force),
we assume that the worker is informal and therefore pays no payroll taxes.

$personal$ is the personal income tax.
We estimate it from ENPH data on $Y$,
and from the number of children in the household.
The ENPH does not allow us to infer which exemptions
(other than the child exemption) people claimed.
However, exemptions are limited and few people claim them.
We are able to see how the tax burden would vary
for the small percentage of individuals who pay income taxes
when we assume that they claim either the minimum or the maximum allowed.

$financial$ is the tax on financial transactions paid by individuals:
\begin{equation}
  financial=0.04 \times (net \;income -350\ uvt)
\end{equation}
where $net \;income$ includes income from all sources net of taxes
and $uvt$ is \textit{unidades de valor tributario}.
\footnote{The tax code uses a special unit to measure income for tax purposes,
called \textit{unidad de valor tributario}, or UVT.
In 2018, the value of a UVT was COP 33,156,
and the monthly minimum wage was COP 781,242.
The tax on financial transactions is only paid
when monthly transactions exceed 350 UVTs,
or around 15 times the minimum wage.}

$wealth$ stands for taxes paid on personal wealth by the individual.
The latest tax on personal wealth expired on December 31, 2018.
As of this writing (Sept 2022), a new one is being mooted.

$property$ is the municipal tax on real estate.
The ENPH reports payments made by individuals on account of this tax,
so we simply add this variable to their tax burden.

$net \; labor \;income$ is labor income reported in the ENPH
minus our estimate of the employee share of payroll taxes
(which is zero for informal workers):
\begin{equation}
net \; labor \;income
  \begin{cases}
	Y - employee \; payroll& Y \geq \underline{w} \\
	Y &  Y < \underline{w}
  \end{cases}
\end{equation}

Finally, $capital \; income$, $dividends$, and $pension \; income$
are self-explanatory, and are reported in the ENPH.

\iffalse
[[SOMEONE:
    What does this next passage mean?
    It looks like it got mangled.
    --jbb]]
\fi

By deciles

- 10.7\% vat, .23 rate on capital
- 10.7\% vat, .15 rate on capital
- 10.7\% vat,  0 rate on capital

By gender and city
Preferred specification: 10.7\% vat and .23 rate on capital

\section{Conclusion}

\citet{microsim}

\iffalse [[ AN ECONOMIST: What in / what document about the microsim
    do we intend to be citing here?
    --jbb ]]
\fi

\begin{appendices}
\section{Building the Data}
The ENPH consists of a number of files
in which the unit of observation is a purchase,
one in which it is a person,
and one in which it is a building.

\subsection{Purchase-level Data}

\subsubsection{Cleaning}

14 files in the ENPH contain what we call ``purchase-level'' data.
It is not literally that,
as it includes within-household transfers (which we drop)
and gifts
and a few taxes (notably the land value tax) --
but ``data on acquisitions and a select set of taxes''
seemed unwieldy.

Every purchase file includes an expense variable.
We drop purchases with a listed cost greater than a billion pesos
(``mil millones de pesos'').

Some of the purchase files include a quantity variable.
For those that don't, we assume quantity = 1.
We drop purchases for which the quantity is non-positive.

Some of the files indicate how the good was acquired.
For those that don't, we assume the good was purchased.

Some of the files indicate where the good was acquired.
For those that don't,
we assume it came from a store too large to dodge the VAT.

Some of the files include a ``frequency of purchase'' variable.
For those that don't, we assume the purchase is monthly.
We drop purchases for which the frequency code is 11, ``never''.

\subsubsection{Incorporating VAT Rates}

Most purchase observations use a COICOP code to indicate what was bought.
However, two of the files, ``Gastos diarios Urbano - Capitulo C'' and
``Gastos semanales Rural - Capitulo C'',
use a separate system of 25 categories to classify certain food purchases.
Based on our study of the tax laws,
we (the Observatorio Fiscal) developed two bridges to VAT rates,
\href
    {https://github.com/ofiscal/tax.co/tree/master/config/vat/grouped/vat\_by\_coicop.csv}
    {one from COICOP codes}
and
\href
    {https://github.com/ofiscal/tax.co/tree/master/config/vat/grouped/vat\_by\_capitulo\_c.csv}
    {one from Capitulo C codes}.

These files include columns ``vat, min'' and ``vat, max''.
That is because the tax laws do not align precisely with the codes the ENPH uses to classify purchases.
For instance, if gasoline for public transportation is not taxed,
but gasoline for private automobiles is,
and if the COICOP system does not distinguish between them,
then the effective tax on fuel depends on the proportion of fuel consumed by buses.

From those legal VAT rates,
we calculate the ``VAT fraction'' for each category as
\begin{equation}
  vatFraction = \frac{vat}{1 + vat}
\end{equation}
We use the VAT fraction to determine
the fraction of a household's expenditures that go to the VAT.
For instance, if the VAT for some good is 50\%, then the VAT constitutes
\begin{equation}
  \frac{0.5}{1.5} = frac{1}{3}
\end{equation}
of the total paid by a consumer.

From VAT rate and purchase frequency, we compute total VAT paid.

Colombia imposes a special 8\% tax on oversized motorcycles.
The size of a motorcycle purchase is not part of the ENPH.
We use price as a proxy, assuming that
any motorcycle with a value above 9 million pesos is oversized.

These tax rates are then merged into the purchase data
by COICOP or Capitulo C code.

Note that the ENPH records the predial (a tax levied on property)
as if it were a purchase, under the COICOP code 12700601.


\subsection{Person-level Data}
Each person in the data is a ``member'' of a household.

Item ``P6050'' encodes an individual's relationship to the head of household
(which might be the identity relationship).
We exclude all persons for whom ``P6050'' takes any of the following values:
\begin{itemize}
\item 6: Empleado(a) del servicio doméstico y sus parientes
\item 7: Pensionista
\item 8: Trabajador
\end{itemize}

\subsubsection{Income}
The item ``P6760'' indicates the number of months
for which someone was paid the last time they received a salary.
Where that value is missing, we assume the received salary is for one month.

The ENPH has more than 70 variables detailing sources of income.
They are mostly missing.
We replace those missing values with zero.
Those columns also use the values 98 and 99, respectively,
to signify that the respondent did not know or would not say, respectively.
(98 and 99 Colombian pesos is an amount of money so close to zero
that nobody would report it as a monthly income of any sort.)
We replace those with 0 as well.

The ENPH provides total cash and in-kind beca and non-beca funding for students.
\iffalse
    [[AN ECONOMIST:
        How can I explain the difference between beca and non-beca funding?]]
\fi
That uses four variables --
cash beca funding, in-kind beca funding, cash non-beca funding, and in-kind non-beca funding.
The ENPH also provides information on
what kinds of entities contributed to a student's funding,
but does not provide a breakdown.
We group those funding sources into two broad classes: government and private.
For those students receiving from multiple agencies,
we assume the contributions are equal across them -- for instance,
if a student receives from one government and two private institutions,
we assume 1/3 of the student's funding is from the government,
and the other 2/3 is private.
The result is 8 variables:
cash x in-kind, beca x non-beca, and government x private.

The ENPH reports 11 kinds of income
an individual might receive from the governmnent.
One of them is unemployment insurance.
The other ten come from the five programs
``Familias en Accion'',
``Familias en Su Tierra'',
``Jovenes en Accion'',
``Programa de Adultos Mayores'', and
``Transferencias por Victimizacion''
in cash and in-kind forms.
We collect all under the category ``government income''.

We define ``non-labor income'' to include beca income and sales income.
The ENPH reports three kinds of sales income --
from stock, from livestock, and from vehicles.

For Colombian tax purposes, ``capital income''
includes rental income and income from interest on investments.
It does not include dividends; we retain that as a separate variable.
Rental income described in the ENPH includes that from
real estate, vehicles and equipment.

We define ``private income'' to include
alimony income, domestic and private remittances, and
assistance (``ayudas'') from domestic and foreign institutions.

We define ``infrequent income'' to include income from
gambling, inheritance, jury awards, tax refunds, and other refunds.
The tax code uses slightly different definitions.
It divides ``ganancias ocasionales'' into
one set that is taxed at a 10\% rate
and another that is taxed at 20\%.
The former includes inheritance income,
assistance from foreign and domestic institutions,
and proceeds from the sale of real estate.
The second includes gambling and jury award income.

The ENPH includes a column that describes ``income'' from borrowing.
We do not include such income in our analysis,
as income is not ordinarily thought of as something one must repay.

Labor income in the ENPH includes all of the following variables:
P6500,
P7070,
P7472S1,
P7422S1,
P6750,
P1653S1A1,
P1653S2A1,
P6585S3A1,
P6585S1A1,
P1653S4A1,
P6510S1,
P6585S2A1,
P1653S3A1,
P6779S1,
P550,
P6630S5A1,
P6630S2A1,
P6630S1A1,
P6630S3A1,
P6630S4A1,
P6630S6A1,
P6590S1,
P6600S1,
P6620S1,
and P6610S1.
Some of those are in-kind and others are cash.
Note that for some rows, some of those columns are redundant,
as certain respondents included their value in their response to P6500.
The variables
P1653S1A2,
P1653S2A2,
P6585S3A2,
P6585S1A2,
P1653S4A2,
P6510S2,
P6585S2A2,
and P1653S3A2 can be used to avoid double-counting them.

The above transformations result in the following income aggregates:
pension,
cesantia,
dividend,
infrequent,
govt cash,
labor cash,
private,
govt in-kind,
and labor in-kind.
From these we are able to compute further aggregates:
cash income,
in-kind income,
government income,
labor income,
and total income.
(Cash and in-kind form a partition of total income.)

\subsubsection {Income Rank and Tax-Deductible Dependents}
To determine which earner should claim which dependent
in households with multiple earners and dependents,
we compute within each household an ``income rank'' from labor income.

The ENPH includes no variable describing disability per se,
but it does provide disability as one of the possible responses (namely 11)
to question P6310, ``Why did you not seek work?''
From this we construct the ``disabled'' variable.

From P6050, the ``relationship to head of household'' variable,
we can assign household members to one of the following categories:
\begin{itemize}
\item head of household
\item relative child
\item relative non-child
\item anyone else
\end{itemize}

A child related to the head of household
can be claimed as a dependent for tax purposes
if they are under 19 years of age,
or if they are a student and under 24 years of age,
or if they are disabled.
A relative who is not a child of the head of household
can be claimed as a dependent
if they have an income below 260 UVTs or they are disabled.
Non-relatives cannot be claimed as dependents.


\subsubsection{Location and Estrato}

The file ``Viviendas y hogares'' contains data on housing.
We use only three variables from it: city, department and estrato.
Estrato is a number assigned to neighborhoods that correlates loosely with income,
ranging from 0 (think hovels with dirt floors and no utilities)
to 6 (pretty swanky).
We merge these variables into the person-level data by household.

\subsection{Taxes on Income}

\subsubsection{The GMF}

The GMF is a financial tax of 0.4\% levied on funds leaving any bank account.
Every individual receives a certain monthly exemption,
which varies from year to year.
The exemption threshold changed at the end of 2016,
halfway through the conduct of the ENPH.
In 2016 it was \$10,413,550 COP, and in 2017 it was \$11,150,650 COP.
We use the arithmetic average of those two figures,
and assume all of a person's cash income leaves their bank account exactly once.

\subsubsection{Other Income Taxes}

Other income taxes levied in Colombia include
Pensión,
Salud,
Solidaridad,
Parafiscales,
Cajas de Compensación,
and Cesantias and Primas.

The rules governing these taxes are somewhat complex,
depending on a person's income and whether they have a long-term employment contract.
In some cases their employer contributes a share.
Our code replicates those laws.
It can be found in the following three files
\href
    {https://github.com/ofiscal/tax.co/tree/master/python/build}
    {here}:
\begin{itemize}
\item people\_4\_income\_taxish.py
\item ss\_functions.py
\item ss\_schedules.py
\end{itemize}


\section{Tax Rules and Modeling Choices}
This appendix describes taxes in Colombia and our modeling choices
with enough precision to replicate our microsimulation.
The tax code we describe here was passed in December 2016,
and it came into force on January 2017.
As of this writing (October 2018),
Congress has already started to discuss a new tax reform.


\subsection{Value Added Tax}
This is a standard VAT (\textit{Impuesto al Valor Agregado}).
Sellers of a good can reclaim the VAT paid for the inputs they used,
and they only have to pay the tax on the value they created.
We assume that this tax is fully borne by consumers.
The survey data from ENPH include consumption data
discriminated by types of goods purchased,
as well as data on income.
We calculate the total VAT paid by an individual
as the value of their consumption of a good
times the VAT rate charged for consuming that good.

The VAT rate is 19\% for all goods and services
unless a different rate is specifically stipulated in the tax code
(\textit{Estatuto Tributario}).
These exceptions are meant to reduce the tax burden
on items that are frequently consumed and are considered basic needs
- they are a way of reducing the regressiveness of the tax.

The exceptions to the standard 19\% VAT fall into one of the following three categories:
\begin{itemize}
\item 5\% VAT on goods listed in article 468 of the tax code
\item Exempt goods and services (\textit{exentos}),
  on which no VAT is paid,
  and for which the seller is allowed to reclaim
  the VAT paid for input purchases.
  These are listed in articles 477-481 of the tax code.
\item Excluded goods and services (\textit{exclu\'{i}dos}),
  on which no VAT is paid,
  but for which the seller is not allowed to reclaim
  the VAT paid for input purchases.
  Excluded goods are listed in article 424 of the tax code,
  and excluded services in article 476.
\end{itemize}

The tax payments made at the 19\%, 5\% and exempt (0\%) rates
are easy to calculate.
Excluded goods, by contrast, require us to make an assumption regarding
the share of the final value that corresponds to
the cost of the inputs used by the final seller.
Our program allows us to choose this share,
which is then multiplied by the final value and then by 19\%
to obtain the VAT paid for this good.

\subsection{Corporate Income Tax}

In principle, there is a flat corporate income tax rate of 33\% on profits,
and in 2018 the marginal tax rate
on profits greater than COP 800 million (roughly USD PPP 626,000)
was provisionally increased to 37\%.
In practice, however, there are numerous industry-specific tax exemptions,
thanks to which these nominal tax rates translate into
vastly different effective tax rates for different firms.
Therefore, we impute an effective tax rate of [[]]
to the earnings of corporations,
which we take from [[]]'s calculation of the effective tax rate on capital,
a figure derived using information from national accounts.

Ideally, we would like to impute the burden of taxes on firms
to the individuals who own the firms.
Unfortunately, this is not possible,
because personal income tax returns do not specify
which firms are the source of a person's earnings from dividends.
Therefore, we impute effective tax rate on capital
to individuals' earnings from dividends and from returns on capital.
These are the taxes paid by the firm before distributing dividends,
and they are not to be confused with
the personal income tax on dividends or on returns on capital,
which is an additional burden to the owners of capital.

\citet{juliana} use administrative data from the tax administration, DIAN,
to analyze high incomes in Colombia.
They show that, in 2010, the top 1\% of income earners
received [[]] percent of their income from dividends
and [[]] from returns on capital.
We impute the effective tax rate on capital to that share of high incomes.

\subsection{Payroll Taxes}

Statutorily, payroll taxes are shared by the employee and the employer.
Because labor demand is elastic and labor supply is highly inelastic,
we attribute the entire burden of payroll taxes to the worker.
Payroll taxes are high in Colombia:
they range between 24\% and 39.5\% of wages
(compare this to the 15.3\% FICA tax in the United States).
Maximum payroll-taxable earnings are 25 times the minimum wage for employees,
and 62.5 times the minimum wage for independent contractors.
We compute the payroll tax burden on the basis of
labor income as reported by respondents to ENPH.
We assume that the income reported is their nominal salary:
it includes the employee share of payroll taxes
but excludes the employer share.
Payroll taxes include:
\begin{itemize}

\item Compulsory Pension Contributions,
  which the employee may choose to put into an individual retirement account
  (\textit{r\'{e}gimen de ahorro individual})
  or into the state-run pay-as-you-go system
  (\textit{r\'{e}gimen de prima media}).
  In general, the employee share is 4\% and the employer share is 12\%.
  However, employees making more than 4 times the minimum wage
  contribute an additional 1\%; this surcharge increases to 1.2\%
  for those making more than 16 times the minimum wage,
  and it gradually goes up to 2\%
  for those making more than 20 times the minimum wage
  \footnote
      {The precise rule is described in
        \textit{decreto n\'{u}mero 510 de 2003}}.
  That is to say, the employee share can be up to 6\% for high salaries.
  Independent contractors pay a different rate than employees:
  only 6.4\% of the value of their contract goes into a pension fund.

\item Compulsory Health Contributions,
  which go to the (usually private) health insurer of one's choice.
  Health insurers must offer a standardized package
  and are paid a fixed amount per patient by the government.
  In general, the employee share is 4\% of wages
  and the employer share is 8.5\%.
  However, employers paying less than 10 times the minimum wage
  do not have to pay the employer share if they are in the for-profit sector.
  Independent contractors pay 5\% of the minimum wage
  or 5\% of the value of their contract
  \footnote{
  Contracts can be signed for less than the minimum wage.
  However, people making less than the minimum wage
  (roughly half the work force) are mostly informal,
  and as such do not make contributions to the health care system},
  whichever is highest, into the health care system.

\item Family Compensation Funds,
  which are chosen by the employer
  and require an employer contribution equal to 4\% of the employee's wages.
  These funds provide services such as recreational centers,
  discounts at supermarkets owned by the funds,
  and in some cases, cash transfers to low-income affiliates.
  Independent contractors are not obligated
  to make payments to the family compensation funds,
  nor can they use their services
  (unless they choose to make payments).

\item Adult Education and Child Welfare Contribution,
  which funds the SENA and ICBF institutes.
  Employers must contribute the equivalent of
  5\% of the employee's salary
  when salaries are higher than 10 times the minimum wage.
  Independent contractors are not responsible for this contribution.

\end{itemize}

\subsection{Personal Income Tax}
Well over 90\% of the population is not required to pay the personal income tax.
This is because the tax only applies to incomes
higher than around 5 times the minimum wage
\footnote{
The tax code uses a special unit to measure income for tax purposes,
called \textit{unidad de valor tributario}, or UVT.
Labor income under 1090 UVTs a year is tax exempt,
and there is a standard deduction equal to 25\% of income.
In 2018, the value of a UVT was COP 33,156,
and the monthly minimum wage was COP 781,242,
so only labor incomes greater than 5.14 times the minimum wage
would potentially pay income tax.},
an amount higher than what most people make.
Married couples are not treated differently from individuals,
and every member of the household who qualifies has to file a tax return.
Deductible income is usually specified as a percentage of gross income,
rather than as a fixed amount in pesos.
The standard deduction is 25\% of income,
and other deductions can add up to 40\% of income.
Income is taxed progressively,
but there are three different schedules
with marginal tax rates that differ depending on the source of income.
Labor income is taxed at marginal rates of 19\%, 28\% and 33\%;
marginal tax rates on income from capital
(e.g., rent payments)
can be 10\%, 20\%, 30\%, 33\%, or 35\%;
and marginal tax rates on income from dividends can be 5\% or 10\%.
Personal income tax is regulated by articles [[]] of the tax code.
The ENPH allows us to distinguish between
labor, capital and dividend income,
and we use this feature of the data to calculate
the personal income tax owed for each type of income.
We assume everyone takes the standard deduction of 25\%.

There is a deduction of [[]]\% for having dependents,
the value of which does not depend on the number of dependents claimed.
We assume households with children minimize their income tax liability
by allocating each dependent to a different earner,
starting with the highest earner and working their way down.

\subsection{Tax on Financial Transactions}

This is a tax of 0.4\% on all financial transactions.
Individuals have a monthly tax-free allowance of around
15 times\footnote{350 UVTs} the monthly minimum wage.
We assume that tax payers optimize so as to pay this tax only once,
so we estimate the tax burden to be
0.4\% of all income higher than the allowance.

\subsection{Import Tariffs}

We assume import tariffs are passed on to consumers of imported goods
and of goods that use imported inputs.
For each individual, we estimate the burden of tariffs to be
\begin{equation}
  \textnormal{Tariff burden} =
      \textnormal{value of consumption}
      \times \textnormal{imports as share of GDP}
      \times \textnormal{average tariff}.
\end{equation}
As of 2017, imports were [[]]\% of GDP,
and the weighted average tariff was [[]]\%.

\subsection{Corporate and Personal Wealth Tax}
The tax reform of 2016 began a phase-out
of the taxes on corporate and personal wealth.
The tax on corporate wealth was last levied in 2017,
and the tax on personal wealth in 2018.

\subsection{National sales Tax}
This is a sales tax (not a VAT) on various items.
We assume that consumers bear the entire burden of
the sales tax
and of the VAT on the inputs used for the production of the final good.
The national sales tax applies mainly to
cell phone, data, and internet services (4\%),
restaurant and bar consumption (8\%),
and vehicles (16\% in most cases).
Most inputs of restaurant consumption are basic food items,
which are tax free,
so for restaurant consumption we calculate a tax of 8\%.
For cell phones, data and internet services,
we assume that inputs amount to a certain share
(60\% unless otherwise noted) of the costs,
so in addition to the sales tax we calculate a tax of
0.6$\times$19\%= 11.4\%.

\subsection{National Fuel Tax}
The national fuel tax is actually
a series of specific taxes on various kinds of fuel.
We convert the specific tax into an ad valorem tax
by dividing it by the market price of a gallon (or cubic meter) of fuel,
and we calculate the amount paid
by multiplying this rate
by a household's expenditure in fuel reported by the ENPH.

\subsection{Stamp Tax}
The stamp tax amounts to only 0.05\% of tax revenue.
We do not model it.

\subsection{Industry and Commerce Tax}
This is a municipal tax on profits.
Municipalities can create exemptions and choose the tax rate,
which by law can be between 0.2\% and 0.7\%
of profits on industrial activities
and between 0.2\% and 1\% of profits on commercial activities.
We account for this tax when we
impute macroeconomic estimates of the tax on capital
to the owners of capital.

\subsection{Property Tax}
This is a municipal tax on property (\textit{impuesto predial}).
The amount paid depends on the cadastral
(as opposed to commercial) value of real estate,
and the rates chosen by the municipality in question.
By law, municipalities can levy a tax anywhere
between 0.1\% and 1.6\% of the cadastral value.
Some municipalities, especially large urban ones,
have up-to-date cadasters in which
the value of real estate approaches its commercial value.
The cadasters of other municipalities,
especially rural ones, are extremely outdated.
This makes it potentially complicated
to estimate the actual burden of the tax.
Fortunately, the ENPH reports households' property tax payments.
We use this variable as our measure of the tax burden,
which we assume falls entirely on the owners of real estate.

\subsection{Municipal and Departmental Gasoline Tax}
These are taxes on gasoline and diesel
that are charged in addition to the national fuel tax.
Municipalities can charge a tax of 18.5\%,
and departments can charge an extra 6.5\%.
In Bogota there is no departmental tax,
but there is a municipal tax of 25\%.
We assume the burden of this tax falls on people who buy fuel
and on people who use public transportation,
in the same way that we do for the national fuel tax.

\subsection{Beer Tax}
\subsection{Liquor Tax}
\subsection{Registry Tax}
\subsection{Tobacco Tax}
\subsection{Vehicle Tax}


\end{appendices}




\bibliographystyle{aea}
\bibliography{references}


\begin{table}
\caption{Taxes in Colombia 2017}
\label{tab:taxcol}
\footnotesize
\begin{tabular}{lccc} \hline
                            & \textbf{\makecell{Billions of \\ USD PPP} }&    \textbf{\makecell{Share of tax \\ revenue}} & \textbf{\makecell{Share of \\ GDP}}      \\     \hline

\textbf{Country Tax Revenue}                            & \textbf{154.9}                & \textbf{100.0\%}         & \textbf{21.3\%}          \\ \hline
\textit{\textbf{National Taxes}}                        & \textit{\textbf{131.5}}       & \textit{\textbf{84.9\%}} & \textit{\textbf{18.1\%}} \\
Value Added Tax                                         & 42.8                          & 27.6\%                   & 5.9\%                    \\
Corporate Income Tax                                    & 39.2                          & 25.3\%                   & 5.4\%                    \\
Payroll Taxes                                           & 25.2                          & 16.2\%                   & 3.5\%                    \\
\hspace{3mm}  \textit{Compulsory Pension Contributions}               & \textit{10.3}                 & \textit{6.7\%}           & \textit{1.4\%}           \\
\hspace{3mm} \textit{Compulsory Health Contributions}                 & \textit{7.5}                  & \textit{4.9\%}           & \textit{1.0\%}           \\
\hspace{3mm} \textit{Family Compensation Fund}                       & \textit{4.9}                  & \textit{3.2\%}           & \textit{0.7\%}           \\
\hspace{3mm} \textit{Adult Education and Child Welfare Contribution} & \textit{2.3}                  & \textit{1.5\%}           & \textit{0.3\%}           \\
Personal Income Tax                                     & 9.2                           & 5.9\%                    & 1.3\%                    \\
Tax on Financial Transactions                           & 5.7                           & 3.7\%                    & 0.8\%                    \\
Import Tariffs                                          & 3.3                           & 2.2\%                    & 0.5\%                    \\
Corporate and Personal Wealth Tax                       & 3.1                           & 2.0\%                    & 0.4\%                    \\
National Sales Tax                                      & 1.6                           & 1.1\%                    & 0.2\%                    \\
National Fuel Tax                                       & 1.3                           & 0.8\%                    & 0.2\%                    \\
Stamp Tax                                               & 0.1                           & 0.0\%                    & 0.0\%                    \\
Other National Taxes                                    & 0.1                           & 0.1\%                    & 0.0\%                    \\ \hline
\textit{\textbf{Municipal Taxes}}                       & \textit{\textbf{17.5}}        & \textit{\textbf{11.3\%}} & \textit{\textbf{2.4\%}}  \\
Industry and Commerce Tax                               & 6.5                           & 4.2\%                    & 0.9\%                    \\
Property Tax                                            & 5.7                           & 3.7\%                    & 0.8\%                    \\
Municipal Gasoline Tax                                  & 1.2                           & 0.8\%                    & 0.2\%                    \\
Other Municipal Taxes                                   & 4.1                           & 2.6\%                    & 0.6\%                    \\ \hline
\textit{\textbf{Departmental Taxes}}                    & \textit{\textbf{5.9}}         & \textit{\textbf{3.8\%}}  & \textit{\textbf{0.8\%}}  \\
Beer Tax                                                & 1.7                           & 1.1\%                    & 0.2\%                    \\
Liquor Tax                                             & 0.9                           & 0.6\%                    & 0.1\%                    \\
Registry Tax                                            & 0.8                           & 0.5\%                    & 0.1\%                    \\
Tobacco Tax                                             & 0.6                           & 0.4\%                    & 0.1\%                    \\
Vehicle Tax                                             & 0.6                           & 0.4\%                    & 0.1\%                    \\
Departmental Gasoline Tax                               & 0.3                           & 0.2\%                    & 0.0\%                    \\
Other Departmental Taxes                                & 1.0                           & 0.6\%                    & 0.1\%                   \\ \hline
\multicolumn{4}{p{14.5cm}}{Data on most national taxes come from DIAN, the national tax agency.
  However, figures for health and pension contributions are taken from the Marco Fiscal de Mediano Plazo; Adult Education and Child Welfare contribution data come from the 2017 national budget law, and Family Compensation Fund figures from the Superintendencia del Subsidio Familiar.
  Data on departmental and municipal taxes are taken from the unified reporting system SISFUT.
  The PPP exchange rate for 2017 is reported by the OECD as COP 1278.039 per USD.} \\
\end{tabular}
\end{table}


\begin{table}[]
\caption{Household Income by gender of head (means unless otherwise stated)}
\label{tab:ygender}
\footnotesize
\begin{tabular}{llll}\hline
&Whole Sample                     & Female-Headed Households & Male-Headed Households                 \\ \hline
Income                           & \$2,131,909              & \$1,757,232            & \$2,393,422     \\
Minimum income                   & \$0                      & \$0                    & \$0             \\
Maximum income                   & \$1,208,333,333          & \$135,267,944          & \$1,208,333,333 \\
Labor Income                     & \$1,573,493              & \$1,226,814            & \$1,815,465     \\
Capital Income                   & \$245,854                & \$188,960              & \$285,565       \\
Dividend Income                  & \$983                    & \$808                  & \$1,105         \\
Pension Income                   & \$212,328                & \$196,620              & \$223,292       \\
Income from Government Transfers & \$14,024                 & \$14,579               & \$13,636        \\
Income from Private Transfers    & \$74,155                 & \$121,061              & \$41,417        \\
Infrequent Income                & \$10,087                 & \$7,581                & \$11,837        \\
Household Size                   & 3.3                      & 3.2                    & 3.4            \\ \hline
\end{tabular}
\end{table}

\begin{landscape}

\begin{table}[]
\caption{Household Income by decile (means unless otherwise stated)}
\label{tab:ydecile}
\footnotesize
\begin{tabular}{lllllllllll} \hline
& Decile 1                              & Decile 2  & Decile 3  & Decile 4  & Decile 5    & Decile 6    & Decile 7    & Decile 8    & Decile 9    & Decile 10                    \\ \hline
Income                           & \$128,967 & \$476,277 & \$727,417 & \$917,718   & \$1,163,786 & \$1,476,187 & \$1,851,703 & \$2,405,887 & \$3,432,354 & \$8,740,423     \\
Minimum income                        & \$0       & \$310,008 & \$612,500 & \$810,098   & \$1,017,154 & \$1,302,533 & \$1,645,229 & \$2,077,000 & \$2,801,608 & \$4,308,000     \\
Maximum income                        & \$310,000 & \$612,500 & \$810,000 & \$1,017,000 & \$1,302,524 & \$1,645,000 & \$2,076,969 & \$2,801,595 & \$4,307,417 & \$1,208,333,333 \\
Labor Income                     & \$69,049  & \$356,369 & \$565,758 & \$758,959   & \$935,016   & \$1,193,366 & \$1,501,921 & \$1,911,513 & \$2,648,808 & \$5,795,531     \\
Capital Income                   & \$8,999   & \$33,206  & \$28,100  & \$36,179    & \$59,813    & \$79,390    & \$107,921   & \$161,095   & \$290,629   & \$1,653,262     \\
Dividend Income                  & \$0       & \$0       & \$0       & \$27        & \$18        & \$19        & \$120       & \$44        & \$44        & \$9,560         \\
Pension Income                   & \$659     & \$7,263   & \$68,695  & \$51,982    & \$87,099    & \$126,031   & \$160,612   & \$239,451   & \$387,066   & \$994,669       \\
Government Transfers & \$15,045  & \$17,247  & \$12,557  & \$13,228    & \$13,851    & \$13,239    & \$11,867    & \$11,444    & \$10,286    & \$21,459        \\
Private Transfers    & \$34,841  & \$61,468  & \$51,613  & \$56,367    & \$66,828    & \$62,307    & \$66,227    & \$77,932    & \$87,849    & \$176,114       \\
Infrequent Income                & \$374     & \$723     & \$693     & \$949       & \$1,143     & \$1,816     & \$2,917     & \$4,362     & \$7,628     & \$80,269        \\
Household Size                   & 2.5       & 2.9       & 3.0       & 3.2         & 3.4         & 3.5         & 3.6         & 3.8         & 3.8         & 3.6             \\
Female-Headed              & 58.4\%    & 49.9\%    & 43.6\%    & 42.2\%      & 42.1\%      & 38.6\%      & 36.1\%      & 35.7\%      & 34.7\%      & 29.6\%         \\ \hline
\end{tabular}
\end{table}

\break

\begin{table}[]
\caption{Household Income by decile (means unless otherwise stated)}
\label{tab:ycity}
\footnotesize
\begin{tabular}{lllllllllll}\\ \hline
& Bogot\'{a}                           & Medell\'{i}n     & Cali          & Bquilla & Bmanga  & C\'{u}cuta       & Tunja        & Cartagena     & S. Marta  & Ibagu\'{e}                     \\ \hline
Income                           & \$3,272,773  & \$2,948,759   & \$2,446,609  & \$2,772,143  & \$2,658,002  & \$1,720,467  & \$2,464,480   & \$2,149,545  & \$1,916,147  & \$2,337,514  \\
Min. income                   & \$0          & \$0           & \$0          & \$0          & \$0          & \$0          & \$0           & \$0          & \$0          & \$0          \\
Max. income                   & \$88,423,500 & \$131,765,257 & \$48,000,000 & \$83,416,765 & \$52,970,667 & \$38,345,000 & \$136,333,350 & \$80,359,167 & \$51,252,500 & \$34,754,167 \\
Labor Income                     & \$2,342,257  & \$2,048,458   & \$1,683,946  & \$2,139,189  & \$1,892,628  & \$1,264,664  & \$1,642,436   & \$1,656,162  & \$1,420,651  & \$1,521,310  \\
Capital Income                   & \$495,082    & \$408,100     & \$259,247    & \$168,671    & \$384,405    & \$181,695    & \$355,207     & \$166,461    & \$99,814     & \$338,837    \\
Dividend Income                  & \$2,890      & \$622         & \$14,289     & \$604        & \$865        & \$2,038      & \$811         & \$0          & \$24         & \$39         \\
Pension Income                   & \$320,524    & \$344,728     & \$335,508    & \$308,685    & \$279,087    & \$138,433    & \$384,296     & \$240,055    & \$314,714    & \$326,645    \\
Govt Transfers & \$9,436      & \$15,006      & \$6,778      & \$11,626     & \$5,497      & \$18,098     & \$6,026       & \$12,627     & \$7,599      & \$14,492     \\
Private Transfers    & \$82,025     & \$94,301      & \$125,328    & \$110,858    & \$71,162     & \$90,099     & \$65,480      & \$72,114     & \$69,067     & \$125,577    \\
Infrequent Income                & \$17,670     & \$36,923      & \$7,224      & \$31,906     & \$23,493     & \$23,403     & \$9,413       & \$2,125      & \$4,255      & \$10,576     \\
Household Size                   & 3.1          & 3.1           & 3.1          & 4.0          & 3.2          & 3.4          & 2.9           & 3.7          & 3.7          & 3.1          \\
Female-Headed  & 39.0\%       & 47.7\%        & 40.9\%       & 37.8\%       & 39.3\%       & 41.8\%       & 45.4\%        & 44.0\%       & 39.7\%       & 46.8\%      \\ \hline
\end{tabular}
\end{table}

\end{landscape}


\begin{table}[]
\caption{Effective tax rate for female and male-headed households}
\label{tab:tgender}
\footnotesize
\begin{tabular}{llll} \hline
&Whole Sample                                          & Female Head & Male Head        \\ \hline
10.7\% rate on consumption              & \$136,200                & \$122,869              & \$143,959   \\
Employee share of health                              & \$62,940                 & \$49,073               & \$72,619    \\
Employee share of pension                             & \$62,940                 & \$49,073               & \$72,619    \\
Pension solidarity                                    & \$0                      & \$0                    & \$0         \\
Employer share of health                              & \$0                      & \$0                    & \$0         \\
Employer share of pension                             & \$188,819                & \$147,218              & \$217,856   \\
Adult education and child welfare                     & \$0                      & \$0                    & \$0         \\
Family compensation fund                              & \$62,940                 & \$49,073               & \$72,619    \\
Tax on financial transactions                         & \$0                      & \$0                    & \$0         \\
Income tax on labor                                   & \$0                      & \$0                    & \$0         \\
10 \% effective tax on capital and dividends          & \$27,426                 & \$21,085               & \$31,852    \\
15 \% effective tax on capital and dividends          & \$43,560                 & \$33,489               & \$50,589    \\
24 \% effective tax on capital and dividends          & \$77,456                 & \$59,548               & \$89,955    \\
Total income                                      & \$2,650,581              & \$2,161,870            & \$2,991,687 \\
Effective tax rate (K 0\%)                  & 19.4\%                   & 19.3\%                 & 19.4\%      \\
Effective tax rate (K 10\%)                 & 20.4\%                   & 20.3\%                 & 20.4\%      \\
Effective tax rate(K 15\%)                 & 21.0\%                   & 20.9\%                 & 21.1\%      \\
Effective tax rate (K 24\%)                 & 22.3\%                   & 22.1\%                 & 22.4\%      \\
                                                      &                          &                        &             \\
Mean Income from Government Transfers                 & \$14,024                 & \$14,579               &       \$16,534      \\
                                                      &                          &                        &             \\
Effective tax rate after government transfers (K24\%) & 21.8\%                   & 21.4\%                 & 21.8\%     \\ \hline
\end{tabular}
\end{table}


\begin{landscape}
\begin{table}[]
\caption{Effective tax rate by income decile}
\label{tab:tdecile}
\footnotesize
\begin{tabular}{lllllllllll} \hline
& Decile 1                                              & Decile 2  & Decile 3  & Decile 4  & Decile 5    & Decile 6    & Decile 7    & Decile 8    & Decile 9    & Decile 10                \\ \hline
10.7\% rate on consumption              & \$50,459  & \$64,555  & \$82,240  & \$98,164    & \$111,867   & \$134,077   & \$157,084   & \$184,478   & \$215,387   & \$287,427    \\
Employee share of health                              & \$0       & \$0       & \$0       & \$0         & \$37,401    & \$47,735    & \$60,077    & \$76,461    & \$105,952   & \$231,821    \\
Employee share of pension                             & \$0       & \$0       & \$0       & \$0         & \$37,401    & \$47,735    & \$60,077    & \$76,461    & \$105,952   & \$231,821    \\
Pension solidarity                                    & \$0       & \$0       & \$0       & \$0         & \$0         & \$0         & \$0         & \$0         & \$0         & \$57,955     \\
Employer share of health                              & \$0       & \$0       & \$0       & \$0         & \$0         & \$0         & \$0         & \$0         & \$0         & \$0          \\
Employer share of pension                             & \$0       & \$0       & \$0       & \$0         & \$112,202   & \$143,204   & \$180,230   & \$229,382   & \$317,857   & \$695,464    \\
Adult ed. and child welfare                     & \$0       & \$0       & \$0       & \$0         & \$0         & \$0         & \$0         & \$0         & \$0         & \$0          \\
Family compensation fund                              & \$0       & \$0       & \$0       & \$0         & \$37,401    & \$47,735    & \$60,077    & \$76,461    & \$105,952   & \$231,821    \\
Tax on financial transactions                         & \$0       & \$0       & \$0       & \$0         & \$0         & \$0         & \$0         & \$0         & \$0         & \$0          \\
Income tax on labor                                   & \$0       & \$0       & \$0       & \$0         & \$0         & \$0         & \$0         & \$0         & \$0         & \$1,105,337  \\
K 10 \%           & \$1,000   & \$3,690   & \$3,122   & \$4,023     & \$6,648     & \$8,823     & \$12,004    & \$17,904    & \$32,297    & \$184,758    \\
K 15 \%           & \$1,588   & \$5,860   & \$4,959   & \$6,389     & \$10,558    & \$14,013    & \$19,066    & \$28,436    & \$51,295    & \$293,439    \\
K 24 \%          & \$2,824   & \$10,420  & \$8,818   & \$11,361    & \$18,775    & \$24,918    & \$33,902    & \$50,564    & \$91,211    & \$521,782    \\
Total income                                          & \$140,792 & \$538,513 & \$826,674 & \$1,050,825 & \$1,477,414 & \$1,876,118 & \$2,354,314 & \$3,045,023 & \$4,316,447 & \$10,601,756 \\
Effective tax rate (K 0\%)                            & 35.8\%    & 12.0\%    & 9.9\%     & 9.3\%       & 22.8\%      & 22.4\%      & 22.0\%      & 21.1\%      & 19.7\%      & 26.8\%       \\
Effective tax rate (K 10\%)                           & 36.5\%    & 12.7\%    & 10.3\%    & 9.7\%       & 23.2\%      & 22.9\%      & 22.5\%      & 21.7\%      & 20.5\%      & 28.5\%       \\
Effective tax rate (K 15\%)                           & 37.0\%    & 13.1\%    & 10.5\%    & 9.9\%       & 23.5\%      & 23.2\%      & 22.8\%      & 22.1\%      & 20.9\%      & 29.6\%       \\
Effective tax rate (K 24\%)                           & 37.8\%    & 13.9\%    & 11.0\%    & 10.4\%      & 24.0\%      & 23.7\%      & 23.4\%      & 22.8\%      & 21.8\%      & 31.7\%       \\
                                                      &           &           &           &             &             &             &             &             &             &              \\
Mean Govt, Transfers                 & \$15,045  & \$17,247  & \$12,557  & \$13,228    & \$13,851    & \$13,239    & \$11,867    & \$11,444    & \$10,286    & \$21,459     \\
                                                      &           &           &           &             &             &             &             &             &             &              \\
Tax rate after govt. (K24\%) & 27.2\%    & 10.7\%    & 9.5\%     & 9.2\%       & 23.1\%      & 23.0\%      & 22.9\%      & 22.4\%      & 21.6\%      & 31.5\%      \\ \hline
\end{tabular}
\end{table}
\end{landscape}


\begin{table}[]
\caption{Effective tax rate for top 1\% (data from \citet{juliana}}
\label{tab:ttop}
\footnotesize
\begin{tabular}{llll} \hline
&Top 1\%                           & Top 0.01\%  & Top 0.001\%                  \\ \hline
10.7\% rate on consumption        & \$591,840                    & \$6,778,439   & \$26,497,772  \\
Employee share of health          & \$234,735                    & \$319,563     & \$353,249     \\
Employee share of pension         & \$234,735                    & \$319,563     & \$353,249     \\
Pension solidarity                & \$58,684                     & \$114,130     & \$105,975     \\
Employer share of health          & \$0                          & \$679,072     & \$750,654     \\
Employer share of pension         & \$704,206                    & \$958,689     & \$1,059,747   \\
Adult education and child welfare & \$0                          & \$399,454     & \$441,561     \\
Family compensation fund          & \$234,735                    & \$319,563     & \$353,249     \\
Tax on financial transactions     & \$0                          & \$0           & \$0           \\
Income tax on labor               & \$7,199,582                  & \$58,572,042  & \$70,942,532  \\
K 10 \%                           & \$1,144,495                  & \$23,799,132  & \$104,315,886 \\
K 15 \%                           & \$1,817,727                  & \$37,798,622  & \$165,678,172 \\
K 24 \%                           & \$3,232,209                  & \$67,211,972  & \$294,602,184 \\
Total income                      & \$21,919,996                 & \$251,053,287 & \$981,398,949 \\
Effective tax rate (K 0\%)        & 42.2\%                       & 27.3\%        & 10.3\%        \\
Effective tax rate (K 10\%)       & 47.5\%                       & 36.7\%        & 20.9\%        \\
Effective tax rate (K 15\%)       & 50.5\%                       & 42.3\%        & 27.2\%        \\
Effective tax rate (K 24\%)       & 57.0\%                       & 54.0\%        & 40.3\%        \\
                                  &                              &               &               \\
Mean Govt. Transfers              & \$21,459                     & \$21,459      & \$21,459      \\
                                  &                              &               &               \\
Tax rate after govt. (K24 \%)     & 56.9\%                       & 54.0\%        & 40.3\%       \\ \hline
\end{tabular}
\end{table}

\begin{landscape}

\begin{table}[]
\caption{Effective tax rates by city}
\label{tab:tcity}
\footnotesize
\begin{tabular}{lllllllllll} \hline
& Bogot\'{a}                            & Medell\'{i}n    & Cali        & Bquilla & Bmanga & C\'{u}cuta      & Tunja       & Cartagena   & S. Marta & Ibagu\'{e}                 \\ \hline
10.7\% rate on consumption        & \$231,224   & \$156,189   & \$160,076    & \$258,109   & \$215,793   & \$142,231   & \$144,688   & \$188,296   & \$164,376   & \$127,111   \\
Employee share of health          & \$93,690    & \$81,938    & \$67,358     & \$85,568    & \$75,705    & \$50,587    & \$65,697    & \$66,246    & \$56,826    & \$60,852    \\
Employee share of pension         & \$93,690    & \$81,938    & \$67,358     & \$85,568    & \$75,705    & \$50,587    & \$65,697    & \$66,246    & \$56,826    & \$60,852    \\
Pension solidarity                & \$0         & \$0         & \$0          & \$0         & \$0         & \$0         & \$0         & \$0         & \$0         & \$0         \\
Employer share of health          & \$0         & \$0         & \$0          & \$0         & \$0         & \$0         & \$0         & \$0         & \$0         & \$0         \\
Employer share of pension         & \$281,071   & \$245,815   & \$202,073    & \$256,703   & \$227,115   & \$151,760   & \$197,092   & \$198,739   & \$170,478   & \$182,557   \\
Adult ed. and child welfare & \$0         & \$0         & \$0          & \$0         & \$0         & \$0         & \$0         & \$0         & \$0         & \$0         \\
Family compensation fund          & \$93,690    & \$81,938    & \$67,358     & \$85,568    & \$75,705    & \$50,587    & \$65,697    & \$66,246    & \$56,826    & \$60,852    \\
Tax on financial transactions     & \$0         & \$0         & \$0          & \$0         & \$0         & \$0         & \$0         & \$0         & \$0         & \$0         \\
Income tax on labor               & \$0         & \$0         & \$0          & \$0         & \$0         & \$0         & \$0         & \$0         & \$0         & \$0         \\
K 10 \%                           & \$55,330    & \$45,414    & \$30,393     & \$18,808    & \$42,808    & \$20,415    & \$39,558    & \$18,496    & \$11,093    & \$37,653    \\
K 15 \%                           & \$87,877    & \$72,127    & \$48,271     & \$29,872    & \$67,989    & \$32,423    & \$62,827    & \$29,376    & \$17,618    & \$59,802    \\
K 24 \%                           & \$156,260   & \$128,254   & \$85,834     & \$53,117    & \$120,895   & \$57,654    & \$111,716   & \$52,234    & \$31,328    & \$106,337   \\
Total income                      & \$4,040,774 & \$3,600,862 & \$2,992,024  & \$3,459,826 & \$3,270,829 & \$2,120,796 & \$3,007,210 & \$2,704,995 & \$2,390,155 & \$2,839,074 \\
Effective tax rate (K 0\%)        & 19.6\%      & 18.0\%      & 18.9\%       & 22.3\%      & 20.5\%      & 21.0\%      & 17.9\%      & 21.7\%      & 21.1\%      & 17.3\%      \\
Effective tax rate (K 10\%)       & 21.0\%      & 19.3\%      & 19.9\%       & 22.8\%      & 21.8\%      & 22.0\%      & 19.2\%      & 22.3\%      & 21.6\%      & 18.7\%      \\
Effective tax rate (K 15\%)       & 21.8\%      & 20.0\%      & 20.5\%       & 23.2\%      & 22.6\%      & 22.5\%      & 20.0\%      & 22.7\%      & 21.9\%      & 19.4\%      \\
Effective tax rate (K 24\%)       & 23.5\%      & 21.6\%      & 21.7\%       & 23.8\%      & 24.2\%      & 23.7\%      & 21.6\%      & 23.6\%      & 22.5\%      & 21.1\%      \\
                                  &             &             &              &             &             &             &             &             &             &             \\
Mean Govt. Transfers              & \$9,436     & \$15,006    & \$6,778      & \$11,626    & \$5,497     & \$18,098    & \$6,026     & \$12,627    & \$7,599     & \$14,492    \\
                                  &             &             &              &             &             &             &             &             &             &             \\
Tax rate after govt. (K24 \%)     & 23.3\%      & 21.1\%      & 21.5\%       & 23.5\%      & 24.0\%      & 22.9\%      & 21.4\%      & 23.1\%      & 22.1\%      & 20.6\%  \\    \hline
\end{tabular}
\end{table}


\begin{table}[]
\caption{Effective tax rates with 17\% VAT}
\label{tab:t17}
\footnotesize
\begin{tabular}{lllllllllll} \hline
\label{tab:t17}

& Decile 1                          & Decile 2  & Decile 3  & Decile 4  & Decile 5    & Decile 6    & Decile 7    & Decile 8    & Decile 9    & Decile 10                 \\ \hline
17\% rate on consumption          & \$75,852  & \$97,042  & \$123,626 & \$147,564   & \$168,162   & \$201,549   & \$236,135   & \$277,314   & \$323,777   & \$432,070    \\
Employee share of health          & \$0       & \$0       & \$0       & \$0         & \$37,401    & \$47,735    & \$60,077    & \$76,461    & \$105,952   & \$231,821    \\
Employee share of pension         & \$0       & \$0       & \$0       & \$0         & \$37,401    & \$47,735    & \$60,077    & \$76,461    & \$105,952   & \$231,821    \\
Pension solidarity                & \$0       & \$0       & \$0       & \$0         & \$0         & \$0         & \$0         & \$0         & \$0         & \$57,955     \\
Employer share of health          & \$0       & \$0       & \$0       & \$0         & \$0         & \$0         & \$0         & \$0         & \$0         & \$0          \\
Employer share of pension         & \$0       & \$0       & \$0       & \$0         & \$112,202   & \$143,204   & \$180,230   & \$229,382   & \$317,857   & \$695,464    \\
Adult education and child welfare & \$0       & \$0       & \$0       & \$0         & \$0         & \$0         & \$0         & \$0         & \$0         & \$0          \\
Family compensation fund          & \$0       & \$0       & \$0       & \$0         & \$37,401    & \$47,735    & \$60,077    & \$76,461    & \$105,952   & \$231,821    \\
Tax on financial transactions     & \$0       & \$0       & \$0       & \$0         & \$0         & \$0         & \$0         & \$0         & \$0         & \$0          \\
Income tax on labor               & \$0       & \$0       & \$0       & \$0         & \$0         & \$0         & \$0         & \$0         & \$0         & \$1,105,337  \\
K 10 \%                           & \$1,000   & \$3,690   & \$3,122   & \$4,023     & \$6,648     & \$8,823     & \$12,004    & \$17,904    & \$32,297    & \$184,758    \\
K 15 \%                           & \$1,588   & \$5,860   & \$4,959   & \$6,389     & \$10,558    & \$14,013    & \$19,066    & \$28,436    & \$51,295    & \$293,439    \\
K 24 \%                           & \$2,824   & \$10,420  & \$8,818   & \$11,361    & \$18,775    & \$24,918    & \$33,902    & \$50,564    & \$91,211    & \$521,782    \\
Total income                      & \$140,792 & \$538,513 & \$826,674 & \$1,050,825 & \$1,477,414 & \$1,876,118 & \$2,354,314 & \$3,045,023 & \$4,316,447 & \$10,601,756 \\
Effective tax rate (K 0\%)        & 53.9\%    & 18.0\%    & 15.0\%    & 14.0\%      & 26.6\%      & 26.0\%      & 25.3\%      & 24.2\%      & 22.2\%      & 28.2\%       \\
Effective tax rate (K 10\%)       & 54.6\%    & 18.7\%    & 15.3\%    & 14.4\%      & 27.0\%      & 26.5\%      & 25.9\%      & 24.8\%      & 23.0\%      & 29.9\%       \\
Effective tax rate (K 15\%)       & 55.0\%    & 19.1\%    & 15.6\%    & 14.7\%      & 27.3\%      & 26.8\%      & 26.2\%      & 25.1\%      & 23.4\%      & 30.9\%       \\
Effective tax rate (K 24\%)       & 55.9\%    & 20.0\%    & 16.0\%    & 15.1\%      & 27.8\%      & 27.3\%      & 26.8\%      & 25.8\%      & 24.3\%      & 33.1\%      \\ \hline
\end{tabular}
\end{table}


\begin{table}[]
\caption{Effective tax rates with 16\% VAT}
\label{tab:t16}
\footnotesize
\begin{tabular}{lllllllllll} \hline
 &Decile 1                          & Decile 2  & Decile 3  & Decile 4  & Decile 5    & Decile 6    & Decile 7    & Decile 8    & Decile 9    & Decile 10          \\ \hline
16\% rate on consumption          & \$72,005  & \$92,121  & \$117,357 & \$140,081   & \$159,634   & \$191,329   & \$224,160   & \$263,252   & \$307,358   & \$410,160    \\
Employee share of health          & \$0       & \$0       & \$0       & \$0         & \$37,401    & \$47,735    & \$60,077    & \$76,461    & \$105,952   & \$231,821    \\
Employee share of pension         & \$0       & \$0       & \$0       & \$0         & \$37,401    & \$47,735    & \$60,077    & \$76,461    & \$105,952   & \$231,821    \\
Pension solidarity                & \$0       & \$0       & \$0       & \$0         & \$0         & \$0         & \$0         & \$0         & \$0         & \$57,955     \\
Employer share of health          & \$0       & \$0       & \$0       & \$0         & \$0         & \$0         & \$0         & \$0         & \$0         & \$0          \\
Employer share of pension         & \$0       & \$0       & \$0       & \$0         & \$112,202   & \$143,204   & \$180,230   & \$229,382   & \$317,857   & \$695,464    \\
Adult education and child welfare & \$0       & \$0       & \$0       & \$0         & \$0         & \$0         & \$0         & \$0         & \$0         & \$0          \\
Family compensation fund          & \$0       & \$0       & \$0       & \$0         & \$37,401    & \$47,735    & \$60,077    & \$76,461    & \$105,952   & \$231,821    \\
Tax on financial transactions     & \$0       & \$0       & \$0       & \$0         & \$0         & \$0         & \$0         & \$0         & \$0         & \$0          \\
Income tax on labor               & \$0       & \$0       & \$0       & \$0         & \$0         & \$0         & \$0         & \$0         & \$0         & \$1,105,337  \\
K 10 \%                           & \$1,000   & \$3,690   & \$3,122   & \$4,023     & \$6,648     & \$8,823     & \$12,004    & \$17,904    & \$32,297    & \$184,758    \\
K 15 \%                           & \$1,588   & \$5,860   & \$4,959   & \$6,389     & \$10,558    & \$14,013    & \$19,066    & \$28,436    & \$51,295    & \$293,439    \\
K 24 \%                           & \$2,824   & \$10,420  & \$8,818   & \$11,361    & \$18,775    & \$24,918    & \$33,902    & \$50,564    & \$91,211    & \$521,782    \\
Total income                      & \$140,792 & \$538,513 & \$826,674 & \$1,050,825 & \$1,477,414 & \$1,876,118 & \$2,354,314 & \$3,045,023 & \$4,316,447 & \$10,601,756 \\
Effective tax rate (K 0\%)        & 51.1\%    & 17.1\%    & 14.2\%    & 13.3\%      & 26.0\%      & 25.5\%      & 24.8\%      & 23.7\%      & 21.8\%      & 28.0\%       \\
Effective tax rate (K 10\%)       & 51.9\%    & 17.8\%    & 14.6\%    & 13.7\%      & 26.4\%      & 25.9\%      & 25.3\%      & 24.3\%      & 22.6\%      & 29.7\%       \\
Effective tax rate (K 15\%)       & 52.3\%    & 18.2\%    & 14.8\%    & 13.9\%      & 26.7\%      & 26.2\%      & 25.6\%      & 24.6\%      & 23.0\%      & 30.7\%       \\
Effective tax rate (K 24\%)       & 53.1\%    & 19.0\%    & 15.3\%    & 14.4\%      & 27.3\%      & 26.8\%      & 26.3\%      & 25.4\%      & 24.0\%      & 32.9\%      \\ \hline
\end{tabular}
\end{table}


\end{landscape}

\begin{table}[]
\label{tab:tcity}
\footnotesize
\begin{tabular}{ll} \hline
&LCR                                             \\ \hline
10.7\% rate on consumption        & \$628,758    \\
Employee share of health          & \$316,000    \\
Employee share of pension         & \$316,000    \\
Pension solidarity                & \$79,000     \\
Employer share of health          & \$671,500    \\
Employer share of pension         & \$948,000    \\
Adult education and child welfare & \$395,000    \\
Family compensation fund          & \$316,000    \\
Tax on financial transactions     & \$0          \\
Income tax on labor               & \$1,129,149  \\
K 10 \%                           & \$0          \\
K 15 \%                           & \$0          \\
K 24 \%                           & \$0          \\
Total income                      & \$11,626,167 \\
Effective tax rate (K 0\%)        & 41.3\%       \\
Effective tax rate (K 10\%)       & 41.3\%       \\
Effective tax rate (K 15\%)       & 41.3\%       \\
Effective tax rate (K 24\%)       & 41.3\%       \\
                                  &              \\
Mean Govt. Transfers              & \$0          \\
                                  &              \\
Tax rate after govt. (K24 \%)     & 41.3\%      \\ \hline
\end{tabular}
\end{table}

\end{document}
